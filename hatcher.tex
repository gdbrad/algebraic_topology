\documentclass[11pt]{article}

\usepackage[euler-digits,small]{eulervm}
\usepackage{bbding} % for flower. 
\usepackage{physics}
\usepackage{amsmath,amssymb}
\usepackage{dsfont}
\usepackage{graphicx}
\usepackage{makeidx}
\usepackage{algpseudocode}
\usepackage{algorithm}
\usepackage{listing}
\usepackage{minted}
%\usepackage{cancel}
% \usepackage{quiver}
%\evensidemargin=0.20in
%\oddsidemargin=0.20in
%\topmargin=0.2in
%\headheight=0.0in
%\headsep=0.0in
%\setlength{\parskip}{0mm}
%\setlength{\parindent}{4mm}
%\setlength{\textwidth}{6.4in}
%\setlength{\textheight}{8.5in}
%\leftmargin -2in
%\setlength{\rightmargin}{-2in}
%\usepackage{epsf}
%\usepackage{url}


\usepackage{booktabs}   %% For formal tables:
                        %% http://ctan.org/pkg/booktabs
\usepackage{subcaption} %% For complex figures with subfigures/subcaptions
                        %% http://ctan.org/pkg/subcaption
\usepackage{enumitem}
%\usepackage{minted}
%\newminted{fortran}{fontsize=\footnotesize}

\usepackage{xargs}
\usepackage[colorinlistoftodos,prependcaption,textsize=tiny]{todonotes}

\usepackage{hyperref}
\hypersetup{
    colorlinks,
    citecolor=blue,
    filecolor=blue,
    linkcolor=blue,
    urlcolor=blue
}

\usepackage{epsfig}
\usepackage{tabularx}
\usepackage{latexsym}
\newcommand\ddfrac[2]{\frac{\displaystyle #1}{\displaystyle #2}}
\newcommand{\N}{\ensuremath{\mathbb{N}}}
\newcommand{\R}{\ensuremath{\mathbb R}}
\newcommand{\coT}{\ensuremath{T^*}}
\newcommand{\Lie}{\ensuremath{\mathfrak{L}}}
\newcommand{\Vectorfield}{\ensuremath{\mathfrak{X}}}
\newcommand{\pushforward}[1]{\ensuremath{{#1}_{\star}}}
\newcommand{\pullback}[1]{\ensuremath{{#1}^{\star}}}
\newcommand{\vectorfield}{\ensuremath{\mathfrak{X}}}

\newcommand{\pushfwd}[1]{\pushforward{#1}}
\newcommand{\pf}[1]{\pushfwd{#1}}

\newcommand{\boldX}{\ensuremath{\mathbf{X}}}
\newcommand{\boldY}{\ensuremath{\mathbf{Y}}}


\newcommand{\G}{\ensuremath{\mathcal{G}}}
% \newcommand{\braket}[2]{\ensuremath{\left\langle #1 \vert #2 \right\rangle}}


\def\qed{$\Box$}
\newtheorem{theorem}{Theorem}
\newtheorem{corollary}[theorem]{Corollary}
\newtheorem{definition}[theorem]{Definition}
\newtheorem{lemma}[theorem]{Lemma}
\newtheorem{observation}[theorem]{Observation}
\newtheorem{proof}[theorem]{Proof}
\newtheorem{remark}[theorem]{Remark}
\newtheorem{example}[theorem]{Example}

\newcommand{\X}{\ensuremath{\mathfrak{X}}}


\newcommand*{\start}[1]{\leavevmode\newline \textbf{#1} }
\newcommand*{\answer}{\start{Answer}}
% \newcommand*{\answer}{\leavevmode\newline \textbf{Answer} }
%\newcommand*{\qed}{\ensuremath{\blacksquare}}

\title{Algebraic topology: Hatcher}
\author{Grant Bradley}


\begin{document}
\maketitle
\tableofcontents
\chapter{Links}
\url{https://pages.uoregon.edu/njp/hw2solutions.pdf} contains solutions to asssigned homework from hatcher.
\url{https://math.berkeley.edu/~hutching/teach/215a-2005/index.html}

Construction of three main gadgets in order to describe a topoological space with algebraic and combinatorial objects:

1. Fundamental group

2. Homology groups

3. Cohomology ring 



\chapter{Ch0}
A deformation retraction is $f_t : X \rightarrow X$ is a special case of the general notion of a \textbf{homotopy} which is any
family of maps $f_t : X \rightarrow Y , t \in I$ s.t. $F: X \times I \rightarrow Y$ given by $F(x,t) = f_t(x)$ is continuous. 



\chapter{Ch1}
\section{Ch1.1}
The fundamental group, a functor, creates an Algebraic image of a space from the loops in the space, paths in the space starting and ending at same point

Can show that two spaces are not homeomorphic by proving that their fundamental groups are not isomorphic ; 
homeomorphic spaces have isomorphic fundamental groups

Induced homomorphisms provide a translation from relations between spaces to relations between their fundamental groups.

Two properties of induced homomorphisms that make the fundamental group a functor:

1. $(\phi\psi)_* = \phi_* \psi_*$ for a composition $(X,x_0)  \underrightarrow{\psi}  (Y,y_0) \underrightarrow{\phi} (Z,z_0)$
\subsection*{Ex2}

2. $\mathds{1}_* = \mathds{1}$, meaning, the id. map $\mathds{1} : X \rightarrow X$ induces the id. map

$\mathds{1} : \pi_1(X,x_0) \rightarrow \pi_1(X,x_0)$







\subsection*{Ex3}

First, suppose that $\pi_1(X,x_1)$ is abelian. Let $f$,$g$ be two loops at the basepoint $x_0$ in $X$. 
$\beta_f(g) = fgf^-1$ , $\beta_g(g) = ggg^-1 \simeq g$. $f(0) = g(0) = x_0$ so $\beta_f = \beta_g$. Thus, $fgf^1 = g$ and 
$fg = gf$ so the group is abelain. 

Prove the converse: assume $\pi_1(X,x_0)$ is nonabelian. Then $[h][f] \not= [f][g]$. If $c$ is a cnst path at the basepoint
$x_0$, $\beta_h(f) = [h][f][h]^{-1} \not= [f] = \beta_c[f]$ . The action of $\beta_f$ will = all $\beta_g$ IFF $f,g$ have
same endpoints

\subsection*{Ex6}

Let $\pi_1(X,x_0)$ as set of basepoint-preserving homotopy classes of maps: $(S^1,s_0) \rightarrow (X,x_0)$, $[S^1,X]$ set of homotopy
classes of maps from $S^1 \rightarrow X$. Show that the natural map $\phi : \pi_1(X,x_0) \rightarrow [S^1,X] $ is onto if $X$ is path
connected; All points contained in $X$ have paths connecting them. 




\subsection*{Ex16}

\subsection*{Ex20}








\subsection{Ex1}
If $f_1 ; g_1 \simeq f_2 ; g_2$ and $g_1 ; g_2$ we must show that $f_1 \simeq f_2$. Let $h: [0, 1] \times S^1 \rightarrow X$
witness $f_1; g_1 \simeq f_2; g_2$ and let $h': [0, 1] \times S^1 \rightarrow X$ witness $g_1 \simeq g_2$. Then
define $h'^{-1}$ as the witness of $g_1^{-1} \simeq g_2^{-1}$. Then say that $f_1 \simeq f_1; g_1; g_1^{-1} \simeq f_2; g_2; \simeq g_2^{-1} \simeq f_2$ via $h; h'^{-1}$.

\subsection{Ex2}
Consider $\beta_h(f) \equiv h \circ f \circ h^{-1}$. If $h \simeq h'$ via the homotopy $H$, then we have that $H; id; H^{-1}$ 
witnessing $h \circ f \circ h^{-1} \simeq h' \circ f \circ h'^{-1}$. Thus, $\beta_h(f) \simeq \beta_{h'}(f)$ for all $f$.

\subsection{Ex3}
%\question{$\pi_1(X)$ is abelian iff $\beta_h$ depend only on endpoints of $h$.}

\start{First part:} Assume $\beta_h$ depends only on endpoints, prove that $\pi_1(X)$ is abelain. Consider $f, g$ loops at $x_0$
See that $\beta_f(g) = f; g; f^{-1}$, and $\beta_g(g) = g; g; g^{-1} \simeq g$. But $f(0) = g(0) = x_0$ and $f(1) = g(1) = x_0$
Hence $\beta_f = \beta_g$, thus $\beta_f(g) = \beta_g(g)$, thus $f; g; f^{-1} = g$, or $f; g = g; h$ thus establishing
that the group is abelian.

\start{Second part:} Assume that $\pi_1(X)$ is abelian. Show that $\beta_*$ depends only on endpoints. Let $f, g$ be two
loops with equal endpoints; $f(0) = g(0)$ and $f(1) = g(1)$. See that:

$$
\beta_f(g) = f; g; f^{-1} = g; f; f^{-1} = g
$$

Thus $\beta_f(g) \simeq g$. So this means that $\beta_f$ will perform no action on any path from $f(0)$ to $f(1)$.
Thus, the action of $\beta_f$ will be equl to all $\beta_g$ as long as $f, g$ have the same endpoints.

\subsection{Ex4}
\start{Idea:} Use compactness to finite finite subcover of star shaped regions. Within each star shaped region,
since star shaped is contractible, all paths are nullhomotopic. So we can homotope the section of the path into a point, and then expand
back into the line segment. [I always get weirded out that homotopy does not preserver cardinality of the image, or indeed any intuition I have about
``structure''].


rigorous seems hard.

\subsection{Ex5}

\end{document}
